%% introduction.tex
%%

%% ==============================
\chapter{Introduction}
\label{ch:Introduction}

\begin{quotation}
"Most domestic energy use, most of the time, is invisible to the user" 
\begin{flushright}
\cite{Darby2006}
\end{flushright}
\end{quotation}

When it comes to energy consumption in domestic households, there is a limited transparency about how much energy is used for different purposes. We only have a vague idea about the benefit of replacing a non-energy efficient household device a with a new energy efficient appliance and what impact a change of daily behaviour has on our energy bill  \citep{Darby2006}.
Just imagine going to a grocery store and buying items without individual price markings. The bill would come on a monthly or even annually basis and would show the accumulated price of all items bought. You would not have any idea, how or when you bought these items or how much they were. Furthermore, you would not have any idea if your consumption was low or high compared to your peers, or how your spendings developed over time  \citep{Kempton1994}.
This lack of timely information prevents  consumers from using energy more intelligently and efficiently\citep{Darby2000}.
Studies have shown that introducing metered energy consumption in domestic households, and providing regular feedback and suggestions, have measurable effects on total energy use and is worth pursuing.  
 \cite{Darby2000} analysed the results of 38 feedback studies and comes to the conclusion, that "direct feedback, almost or in combination with other factors, is the most promising single type". Direct feedback is available on demand and users can access information about their energy usage via direct displays, smart meters and interactive feedback. The learning approach is in a "looking or paying" sense, since direct feedback requires an active attitude from the costumer.
Some of the surveyed studies showed that direct feedback in combination with some form of advice or information enabled savings up to 10\%.
Smart meters are the most prominent example of providing direct feedback to household. The value of the smart meters' home installation is that each unit can tell households on demand how high the energy consumption is at the moment and can add additional information like the consumption of individual appliances and the efficiency of installed devices compared to a potential new one. The household's benefit from the information provided by smart meters, however, depends on whether the information is tangible to the user.
At this point, we introduce the thesis for our research paper:
\begin{quotation}
What information do consumers need to know about the objects they own or manage to increase their energy efficiency? 
\end{quotation}

\dots