%% introduction.tex
%%

%% ==============================
\chapter{Introduction}
\label{ch:Introduction}

\begin{quotation}
"Most domestic energy use, most of the time, is invisible to the user." 
\begin{flushright}
\cite{Darby2006}
\end{flushright}
\end{quotation}

When it comes to energy consumption in domestic households, there is a limited understanding about how much energy is used for different purposes - we only have a vague idea about how much energy we consume during our daily routines. Just compare the energy consumption with the weekly trip to the grocery store. In a normal grocery store, every item has its price tag, offering full transparency about what we pay to use it. Now imagine a grocery store, where you can buy the energy usage of the fridge, the room lightings etc. None of these offered items would indicate an individual price marking. The bill is not presented directly, but would come on a monthly or even annually basis and would show the accumulated price of all items bought. You would not have any idea how or when you bought these items or how much they were individually. Furthermore, you would not have any idea if your consumption was low or high compared to your peers, or how your spendings developed over time  \citep{Kempton1994}. As a result, we only get a limited understanding about what impact a change of daily behaviour has on our energy bill  \citep{Darby2006}. In addition a potential benefit of replacing a non-energy efficient household device with a new energy efficient appliance is unknown.\\
This lack of timely information prevents consumers from using energy more intelligently and efficiently \citep{Darby2000}.
Studies have shown that introducing metered energy consumption in domestic households, and providing regular feedback and suggestions, have measurable effects on total energy use and is worth pursuing.  
Darby analysed the results of 38 feedback studies and came to the conclusion, that "direct feedback, almost or in combination with other factors, is the most promising single type". Direct feedback is available on demand and users can access information about their energy usage via direct displays, smart meters and interactive feedback. The learning approach is in a "looking or paying" sense, since direct feedback requires an active attitude from the customer.
Some of the surveyed studies showed that direct feedback in combination with some form of advice or information enabled a cost savings of up to 10\%.\\
Smart meters are the most prominent example of providing direct feedback to households. The value of the smart meters' home installation is that each unit can tell households, on demand, how high the energy consumption is at the moment. Moreover, it can add additional information like the consumption of individual appliances and the efficiency of installed devices compared to a potential new one. The household's benefit from the information provided by smart meters, however, depends on whether the information is tangible to the user.
Therefore, we define the research question for this thesis:
\begin{quotation}
How much information do consumers need to know about the objects they own or manage to increase their energy efficiency? 
\end{quotation}
We try to define a level of information provided by smart meters that is sufficient for users to help them reduce their energy consumption, but does not overburden them with too much information, resulting in the danger of an information overload.\\
In order to find the optimal level of information load in a smart meter environment, we design a behavioural experiment that serves as a proxy for a situation in which a user of a smart meter faces the following task: the user must choose out of several options in order to reduce his or her energy consumption. Thereby, the optimal choice is influenced by two driving factors, the personal benefit and the cost of the option (e.g. kWh).\\
In our experiment, we transfer this choice environment into a knapsack-problem. In this type of optimization problem, one must choose between several items that show two characteristics: a benefit and a cost. The goal is to maximize the cumulated benefit of all chosen items without extending a given cost restraint.\\ 
This thesis is a first step to design this experiment. We abstract from the energy context and introduces a plain version of the experiment. So we do not refer to energy cost and appliance's benefit when referring to cost and benefit, but try to identify the driving forces behind the experiment without the affiliation with the energy context. A second step will then use the results from this experiment and transfer the experiment to an energy context.\\ 
By finding the optimal level of information detail for smart meters, we can contribute to develop smart meters that will help consumers to reduce their energy consumption in the future.\\ 
The remainder of this thesis is organized as follows. Section 2 gives a brief overview of the underlying literature, describing the current research about smart meters and provides scientific background about the phenomenon "information overload". Afterwards, we define our main hypotheses.
Section 3 introduces the used experiment and gives details about the descriptives of the gained data. Section 4 evaluates the outcomes of the experiment and tests the hypotheses. Finally, Section 5 concludes the findings of the experiment and gives an outlook about potential fields of future research.