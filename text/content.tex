%% content.tex
%%

%% ==============
\chapter{Literature Review}
\label{ch:Literature Review}


%%% ==============

%%% ===========================
%\section{Feedback}
%\label{ch:Literature Reviewk:sec:Feedback}
%%% ===========================
%

% 
%Studies about how to increase the transparency of energy consumption date back as far as the 1970s \cite{Darby2006}. Early studies saw feedback as a behaviour reinforcement, incentivizing individuals who were seen passive and motivated\footnote{One study used notes posted on the kitchen window to inform the participants about his or her previous day's consumption compared with some reference level.}.
%Later research concentrated  on providing feedback within a system context, instead of aiming for target behaviours. 
%
%
%
%psychologists tried to intervene in the
%
% found that feedback on energy consumption has measurable effects and was worth pursuing.
% 
%Feedback
%
%Studies have shown, that implementing and improving feedback on energy consumption has many benefits for the consumer. 
%Feedback... 
%\begin{itemize}
%\item helps the consumer learn more about how to control energy use more effectively over a long period of time;
%\item is needed, in form of instantaneous direct feedback  in combination with frequent, accurate billing  as a basis for sustained energy reduction;
%\item is useful on its own, as a self-teaching tool;
%\item improves the effectiveness of other information:
%\item advices in achieving better understanding and control of energy use
%\item can lead to energy reduction in non-covered devices, the so-called "Spill-Over-Effect SOURCE UENO.
%\end{itemize}
%
%As a result, feedback plays a major role in raising energy awareness and helping consumers reducing their energy consumption up to 10% \cite{Darby2000} .
%
%Two main forms of feedback can be identified:: direct feedback in the home and indirect feedback via billing. \cite{Darby2000}
%\paragraph{Indirect Feedback}
%Indirect feedback processes raw data processed by the utility and sent out to consumers.This includes frequent billing with information about energy usage and additional information, such as historical comparison or audit offers.
%consumers are supposed to learn about their consumption by reading and reflecting.  
%\paragraph{Direct feedback}
%Unlike indirect feedback, direct feedback is available on demand. Users can access information about their energy usage for example via direct displays, smart meters and interactive feedback. The learning approach is more in a 'looking or paying' sense, since direct feedback requires a more active attitude from the consumer.
%

% 
%\begin{tabular}{|c|c|c|}
%\hline 
%Differences & Indirect Feedback & Direct Feedback \\ 
%\hline 
%• & • & • \\ 
%\hline 
%• & • & • \\ 
%\hline 
%• & • & • \\ 
%\hline 
%• & • & • \\ 
%\hline 
%\end{tabular} 

The growing importance of information feedback on energy consumption is reflected in the growing number of literature covering this specific topic. Research on direct feedback, in the form of smart meter technology, examines the advantages and disadvantages of the technology and offers important recommendations to policy makers, businesses and homeowners.

%% ===========================
\section{Smart Meter}
\label{ch:Content1:sec:Smart Meter}
%% ===========================
\citep{Darby2008} defines the smart meter as 
\begin{quotation}
\textit{a meter that stores information and gives accurate consumption data\\ 
at specified intervals to suppliers and consumers.}
\end{quotation}

The full potential of smart meters enables benefits for consumers, suppliers and regulators by defining a new level of communication between these parties.  This can lead to a positive impact on overall consumption, on load management and on consumer retention.

Benefits for consumers include access to on-demand information and, in combination with user-friendly feedback, the opportunity to reduce the energy bill.
Suppliers can benefit from better demand-management by providing incentives for customers to reduce peak-time consumption or to save the expenses of manual meter reading. Smart meters also open the market to a "smart home" environment, which includes a remote energy management system for individuals to control aspects such as lights and heating.
This new relationship between suppliers and consumers is not only beneficial to consumers and suppliers, but can also support regulators in their pursuit of reducing carbon emissions  \citep{Darby2008}.

\paragraph{Potential Information provided to consumers}

The data provided by the smart meter can cover a wide range of different aspects\footnote{Refer to \citep{WillAnderson2009}.} of household energy consumption such as:
    \begin{itemize}
    \item Power consumptions of individual devices
    \item Energy costs of individual devices
    \item Household baseload consumption
    \item Range of household's energy consumption levels
    \item Individual definition of a high level of consumption
    \item Typical daily consumption or spending
    \item Patterns of energy consumption over the week
    \item Link between individual and collective energy consumption
    \end{itemize}%
Even though the smart meter offers a potential variety of information, the data still needs to be intelligible to users. According to \cite{WillAnderson2009}, an ideal smart meter does not only have to make energy usage visible, but should also capture the user's interest by providing useful information while minimizing data that might be deemed too detailed or too complex.

Previous smart meter experiments support Anderson's hypothesis. They came to the conclusion that the potential for receiving energy feedback depends on the capability of the user to understand and process the given information.
Feedback that contains too much information can overwhelm the user and reduce his or her capability to process the information \citep{Henryson2000}. Consequently,  adding information or tools may complicate a decision rather than making it easier \citep{Darby2006} and users might face an \textit{information overload} \citep{Fischer2008}. Therefore, \cite{WillAnderson2009} underline the necessity to further research the question, how can an information overload be avoided in a smart meter environment?

%% ===========================
\section{Information Overload}
\label{ch:Literature Review:sec:Information Overload}
%% ===========================

Research on information overload has a long tradition in marketing science.
The basic idea behind information overload is that people tend to make poorer and less effective decisions when being presented with too much information at any given time \citep{Siegfried1965}. 
This results in the question, how much information is too much for individuals?\\
Before describing different levels of information load, one must first define how to find an accurate measure of information load, e.g. to make the outcome of different experiments comparable.
\subsection{Measure of information load}
The traditional approach measures the information load by counting the number of alternatives and attributes presented to the consumer \citep{Chen2009}. This two-dimensional approach is criticized in the later literature and new measuring techniques of information have evolved. Information load was now not only considered as a product of alternatives and attributes, but as being influenced by many factors, including the information structure \citep{Lurie2004}, information quality \citep{Keller1987}, time pressure, the diversity of information dimension \citep{Payne1982} and information repetitiveness \citep{Hwang1999}.
More recently, the distribution of attributes and alternatives has been added into the pool of factors which might influence the information load \citep{Lurie2004}. 
All in all, the issue of how to measure information load is a highly debated topic in the literature. And even though the magnitude of the influence from both attributes and alternatives is still a focus of current research, these factors are still commonly used as a base for information load experiments.

\subsection{Impact of information overload}

Since there is no common academic ground about how to define information load, the research on the impact of an information overload returns mixed and inconsistent results.
One of the first main results from studies conducted by the traditional approach is the inverted U-shaped relationship between the amount of information and the choice accuracy \citep{Jacoby1974}. Jacoby concluded that there is an optimal information load for consumers to make the best and most effective decision and consumers \textit{can} be overloaded with information. Nevertheless, individuals \textit{will not} be overloaded according to Jacoby since they "are highly selective in how much and just which information they access". Thus, facing an information overload leads individuals to focus selectively on the information they feel is important.
Later studies, however, have questioned Jacoby's thesis, in particular the U-shape (e.g. \cite{Malhotra1982}), since his results could not be re-produced. Nevertheless, an impact of information load on choice accuracy was still detected by the majority of experiments.

\section{Information Overload in a knapsack problem}
\label{ch:Literature Review:sec:Information Overload in a knapsack problem}
In this paper, we compare the situation where a consumer needs to make an efficient decision on which of the appliances belongs to an energy-efficient portfolio with the knapsack problem. The knapsack problem describes an optimization problem in which a person chooses out of a set of items with individual weights and benefits. The aim of the optimization is to maximize the cumulated benefit of all the chosen items while not exceeding a given weight restriction.

By transferring the abstract level of the knapsack problem to an energy efficiency context, one can imagine that household appliances have a benefit and weight for a consumer. The benefit of the appliance is how good the appliance meets the requirements of its owner, and the weight could be its energy consumption. The optimization problem in this context is to pick the best selection of appliances that both meet the personal requirements and do not exceed an individual budget. This can include replacing or eliminating an existing appliance.
The information overload in a knapsack problem is produced by a high number of items to choose from, ergo a high number of alternatives, and a constant number of attributes, weight and benefit. The parameter that is manipulated is the number of attribute levels, the so-called information granularity.

\section{Hypotheses}
\label{ch:Literature Review:sec:Hypotheses}
The experiment evaluates the choice accuracy of the participants, the time it takes them to make a decision and the time it taks an individual to complete one round of the experiment.
Choice accuracy is measured as how close the cumulated benefit of the knapsack is to the optimal solution of the optimization. The closer the total benefit of the knapsack gets to the optimal solution, the better the choice accuracy.
In order to analyse the choice accuracy in a knapsack setting, we consider three different points in time. First, the total benefit of the first solution (i.e. the first time the knapsack is filled and no further items can be added). Second, the total benefit of the best solution (i.e. the set with the highest cumulated value). Third, the total benefit of the final solution (i.e. the last set represented in the knapsack). Furthermore, we evaluate the number of decisions made in one round and the average time it took a participant to make a decision.


%\subsection{Information structure}
%The first research question to be explored by this paper is the extent to which participants respond to different information structures.
%\cite{Lurie2004} observes in his study that information structure affects the information overload. He argues that structural elements, such as the number of attribute levels, can have an impact on the choice accuracy.
%In the experiment, the information is structured as boxes that represent the items - the width of the boxes represents its weight; every box has the same height. Thus, the wider the box, the greater the weight. 
%The information about the benefit of the item was presented in two structural approaches.
%One half of the usergroups is presented boxes whose colours are a representation of the ratio of benefit and weight of the item. The other half are presented boxes that are coloured according to the benefit of the box.\\ 
%We use the ratio approach to make the decisions as simple as possible to the participant and therefore try to isolate the complexity of the game to the information load. By doing that, we take away one major challenge for the knapsack optimization, computing the benefit-to-weight-ratios to make boxes comparable and consequently reduce the information per box to the width and the benefit-to-cost ratio. According to Lurie, reducing the information per element will then lead to a increasing choice accuracy\footnote{Lurie talks about an increasing decision quality what corresponds to choice accuracy.} and less time spent per decision.\\
%In order to detect a possible oversimplification by installing the ratio approach, we created a control group that follows the traditional setup from the knapsack - showing weight and a colour according to the benefit.
%Following the results from Lurie, we argue that the the choice accuracy will be better for the usergroups with a ratio setup and participants will make quicker decision in the ratio setup.
%
%\textit{Hypothesis 1 (H1a-H1c): Keeping the information granularity constant, the total benefit of the first(a), best(b), final(c) solution is higher when the colour of the box represents the benefit-weight-ratio. }
%
%\textit{Hypothesis 2 (H2): Keeping the information granularity constant, the average amount of time per decision is lower for the benefit-weight-ratio. }


\paragraph{Information granularity}
The goal of this thesis is to answer how participants respond to different levels of information granularity.
Information granularity in the experiment is designed as the number of colours representing the benefit. Two or three different colours are considered a low level of information granularity, seven colours are a mediate level and eleven and fifteen colours are defined as a high level of information granularity.
For the experiment we follow the hypothesis of \cite{Jacoby1974} which forms an inverted U-shape relation between information load and choice accuracy. Consequently, we argue that a mediate level of information load will have the best choice accuracy.\\
\textit{Hypothesis 1 (H1a-H1c): The total benefit of the first(a), best(b), final(c) solution is best on a mediate level of information detail.}

\cite{Jacoby1974} indicate that the relationship of the information granularity\footnote{Jacoby et al. refer to \cite{Hendrick1968} who defined complexity by the number of different dimensions of an attribute. In our case, this corresponds to the number of possible colours per item, our definition of information granularity. }  is curvilinearly correlated to the time it takes a person to make a decision. From a low level to a mediate level of information granularity, the time spent on one decision increases with the information load. After a specific level of information granularity, a decision becomes more and more complex \citep{Hendrick1968}. Therefore, individuals tend to give up trying to compare alternatives, and instead make their choices impulsively. In doing this, individuals simplify their information processing by ignoring some portion of the provided information \citep{Malhotra1982}.
As a result, the time it takes to make an individual decision is high on a mediate level of information detail. However, the number of decisions is the lowest for the medium level since the choice is more accurate. Therefore an individual concentrates on the quality of the choice rather than by the quantity.\\ 
\textit{Hypothesis 2 (H2): The average amount of time per decision is the highest on a mediate level of information detail.}\\
\textit{Hypothesis 3 (H3): The number of decisions is the lowest on a mediate level of information detail.}\\

\paragraph{Learning effect}
The second parameter to evaluate is whether participants learn from playing the game and can adapt to an information overload. \cite{Jacoby1974} indicates that individuals develop the ability to accommodate large amounts of information. Thus, we expect participants to learn to cope with similar information loads the more they experience it. In the experiment, participant play 3 rounds of the knapsack problem with a constant information load. We therefore argue that participants will improve their choice accuracy with each round.\\
\textit{Hypothesis 4 (H4a-H4c): The first(a), best(b), final(c) solution increases with the number of repetitions of the task. }\\
In addition, we expect participants to make quicker decisions with a growing experience, and a decreasing amount of decisions since the quality of the decisions increase and therefore fewer decisions are necessary. \\
\textit{Hypothesis 5 (H5): The average amount of time per decision decreases with the number of repetitions of the task. }\\
\textit{Hypothesis 6 (H6): The number of decisions decreases with the number of repetitions of the task. }\\
Since the total time is the product of the average decision time and the number of decisions, we argue that the total time it takes participants to reach a solution decreases with the amount of rounds played.\\
\textit{Hypothesis 7 (H7a-H7c): The first(a), best(b), final(c) solution is reached faster with the number of repetitions of the task. }

\begin{table}[htbp] % Usergroup and Hypotheses
  \centering
  \caption{Treatments and Hypotheses}
  \label{tab:Hypotheses}
    \begin{tabular}{cc|rrr}
    \toprule
    \textbf{Treatment} & \textbf{\# Colours} & \multicolumn{3}{c}{\textbf{Hypotheses}} \\
    \midrule
    \textbf{1} & 2     &  & \multicolumn{1}{l}{H1a-c: } & \multicolumn{1}{l}{Treatment 1+2 < 3 > 4+5} \\
    \textbf{2} & 3     &       & \multicolumn{1}{l}{H2: } & \multicolumn{1}{l}{Treatment 1+2 < 3 > 4+5} \\
    \textbf{3} & 7    &    &  \multicolumn{1}{l}{H3: } & \multicolumn{1}{l}{Treatment 1+2 > 3 < 4+5} \\
    \textbf{4} & 11    &    &  \multicolumn{1}{l}{H4a-c: } & \multicolumn{1}{l}{Round 1 < 2 < 3} \\
    \textbf{5} &  15	   &    &  \multicolumn{1}{l}{H5: } & \multicolumn{1}{l}{Round 1 > 2 > 3} \\
     & 		&    &  \multicolumn{1}{l}{H6: } & \multicolumn{1}{l}{Round 1 > 2 > 3} \\
     & 		&    &  \multicolumn{1}{l}{H7a-c: } & \multicolumn{1}{l}{Round 1 > 2 > 3} \\
    \bottomrule
    \end{tabular}%
\end{table}%
%
%
%Questions
%More recently, \cite{Chen2009} found that online environment overload results "in less satisfied, less confident, and
%more confused consumers".
%
%
%How mentally demanding was the task?
%How hurried or rushed was the pace of the task?
%How successful were you in accomplishing what you were asked to do?
%How hard did you have to work to accomplish your level of performance?
%How insecure, discouraged, irritated, stressed, and annoyed were you?
%What box attribute did you mainly look for to reach your result?
